\documentclass[10pt,pdf,hyperref={unicode}]{beamer}
\usepackage{amsmath, amsfonts}
\usepackage[utf8]{inputenc}
\usepackage[russian]{babel}
\usepackage{graphicx}
\usetheme{Warsaw}

\DeclareMathOperator{\svert}{\,\vert\,}
 
\setbeamertemplate{footline}[page number]{}
\setbeamertemplate{navigation symbols}{}

\hypersetup{pdfpagemode=FullScreen}
\author{Ефим Пышнограев}
\date{22 апреля 2014}
\institute{Филиал МГУ в городе Севастополе}
\title{Извлечение описаний событий предопределённых типов из потока сообщений пользователей микроблогов}

\begin{document}

\begin{frame}
  \titlepage
\end{frame}

\begin{frame}
  \frametitle{Постановка задачи}
  Данные:
  \begin{itemize}
  	\item 
  	множество документов (сообщений) 
  	$\Omega = \left\{D_i \svert i \in \overline{1,n} \right\},$
	\item 
	документ --- множество слов и временная метка $t_i$ 
	$D_i = \left\{w_j \svert j \in \overline{ 1, l_i } \right\}.$
  \end{itemize}
  
  События:
  \begin{itemize}
	\item резкое увеличение частоты ключевых слов, затем спад до нормального уровня,
	\item должно быть вызвано реальным событием,
	\item не носит периодический характер.
  \end{itemize}
  
  Особенности сети Твиттер:
  \begin{itemize}
  \item короткие сообщения (до 140 символов),
  \item наличие шума и ошибок,
  \item большая плотность сообщений,
  \item ``взрывной'' характер событий.
  \end{itemize}
  
\end{frame}

\begin{frame}
  \frametitle{Схема алгоритма}
  Составные части:
  \begin{itemize}
  \item нахождение максимума в частотной функции, который будет соответствовать событию,
  \item извлечение ключевых слов, характерных этому максимуму,
  \item применение модели HDP с частичным обучением для того чтобы выделить все сообщения этой темы,
  \item проверка насколько полученный результат соответствует новому событию.
  \end{itemize}
\end{frame}

\begin{frame}
  \frametitle{Результаты}
\end{frame}

\begin{frame}
  \frametitle{Заключение}
\end{frame}

\end{document}
