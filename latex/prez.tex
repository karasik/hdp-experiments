\documentclass[10pt,pdf,hyperref={unicode}]{beamer}
\usepackage{amsmath, amsfonts}
\usepackage[utf8]{inputenc}
\usepackage[russian]{babel}
\usepackage{graphicx}
\usepackage{animate}
\usepackage{comment}
\usetheme{Warsaw}
 
\makeatletter
\defbeamertemplate*{footline}{my theme}{
    \leavevmode%
    \hbox{%
    \begin{beamercolorbox}[wd=.2\paperwidth,ht=2.25ex,dp=1ex,center]{author in head/foot}%
        \usebeamerfont{author in head/foot}%
        \insertauthor
    \end{beamercolorbox}%
    \begin{beamercolorbox}[wd=.7\paperwidth,ht=2.25ex,dp=1ex,center]{title in head/foot}%
        \usebeamerfont{title in head/foot}\inserttitle
    \end{beamercolorbox}%
    \begin{beamercolorbox}[wd=.1\paperwidth,ht=2.25ex,dp=1ex,right]{date in head/foot}%
        \usebeamerfont{date in head/foot}
        \insertframenumber{} / \inserttotalframenumber\hspace*{2ex}
    \end{beamercolorbox}}%
}
\makeatother
\beamertemplatenavigationsymbolsempty 

\hypersetup{pdfpagemode=FullScreen}
\newcommand{\const}{\ensuremath{const}}
\author{Ефим Пышнограев}
\date{24 мая 2013}
\institute{Московский государственный университет имени М. В. Ломоносова}
\title{Решение задачи диффузии-конвекции в жидкости с пульсирующим источником}

\begin{document}
\begin{frame}
  \titlepage
\end{frame}

\begin{frame}
  \frametitle{Введение}
  \begin{itemize}
    \item Диффузия вещества в жидкости
    \item Какую модель выбрать?
  \end{itemize}
\end{frame}

\begin{frame}
  \frametitle{Введение}
  \begin{itemize}
    \item Простейшая модель
      \begin{equation*}
        u_t = a^2 u_{xx}
      \end{equation*}
      \begin{figure}[ht]
        \begin{center}
          \includegraphics[width=5cm]{int1.eps}
        \end{center}
      \end{figure}
  \end{itemize}
  Что можно добавить в модель чтобы сделать ее достовернее?
\end{frame}

\begin{frame}
  \frametitle{Введение}
  Что можно добавить в модель чтобы сделать ее достовернее?
  \begin{columns}
    \column{0.5 \textwidth}
    \begin{itemize}
      \item Добавим конвективное слагаемое
    \end{itemize}
    \begin{equation*}
      u_t = a^2 u_{xx} + w u_x
    \end{equation*}
      \begin{figure}[ht]
        \begin{center}
          \includegraphics[width=3cm]{int2.eps}
        \end{center}
      \end{figure}

    \column{0.5 \textwidth}
    \begin{itemize}
      \item Сделаем область многослойной
    \end{itemize}
    \begin{equation*}
      u_t = a(x) u_{xx}
    \end{equation*}
      \begin{figure}[ht]
        \begin{center}
          \includegraphics[width=3cm]{int3.eps}
        \end{center}
      \end{figure}
  \end{columns}
\end{frame}

\begin{frame}
  \frametitle{Постановка задачи}
  Остановимся на этих двух уточнениях.
  \begin{columns}
    \column{0.6 \textwidth}
  \begin{figure}[ht]
    \begin{center}
      \includegraphics[width=5cm]{int4.eps}
    \end{center}
  \end{figure}
    \column{0.4 \textwidth}
  \begin{equation*}
   u_t = wu_x + a^2 u_{xx},
  \end{equation*}
  \begin{equation*}
   u(x,0) = 0,
  \end{equation*}
  \begin{equation*}
    \left\{
    \begin{aligned}
      & u(-l_1,t) = 1 - \cos \omega t, \\
      & u_x(l_2,t) = 0, \\
      & u(-0, t) = u(+0, t), \\
      & a_1^2 u_{x}(-0, t) = a_2^2 u_{x}(+0, t). \\
    \end{aligned}
    \right.
  \end{equation*}
  \end{columns}
\end{frame}

\begin{comment}
\begin{frame}
  \frametitle{Возможные подходы}
  \begin{columns}
    \column{0.5 \textwidth}
    Численный метод
    \begin{itemize}
      \item Применим в большинстве случаев, обычно не требует сложных предварительных выкладок
      \item Неизвестен общий вид искомой функции
      \item Часто необходимо много вычислительных ресурсов
    \end{itemize}

    \column{0.5 \textwidth}
    Аналитический метод
    \begin{itemize}
      \item Может быть использован не всегда, при решении часто бывают трудности
      \item Если все получилось, то получаем явную формулу для результата
      \item Точное решение может использоваться для проверки адекватности численного метода
    \end{itemize}
  \end{columns}
\end{frame}
\end{comment}

\begin{frame}
  \frametitle{Этапы решения}
  Решение задачи будем находить методом конечных интегральных преобразований:
  \begin{itemize}
    \item Переход к однородным граничным условиям
    \item Разделение переменных
    \item Составление задачи Штурма--Лиувилля
    \item {\bf Вывод уравнения для собственных чисел}
    \item {\bf Нахождение общего вида собственных функций}
    \item Вычисление веса, с которым ортогональны собственные функции
    \item Проведение интегрального преобразования
    \item {\bf Нахождение временной составляющей решения}
    \item {\bf Получение окончательного ответа}
  \end{itemize}
\end{frame}

\begin{comment}
\begin{frame}
  \frametitle{Переход к однородным граничным условиям} 
  Производим замену
  \begin{equation*}
    u(x,t) = U(x,t) + (1 - \cos \omega t).
  \end{equation*}
  В результате приходим к новому уравнению, начальному и граничным условиям
  \begin{equation*}
    U_t(x,t) = w U_x(x,t) + a^2 U_{xx}(x,t) - \omega\sin\omega t,
  \end{equation*}
  \begin{equation*}
    U(x,0) = 0,
  \end{equation*}
  \begin{equation*}
    \left\{
    \begin{aligned}
      & U(-l_1,t) = 0, \\
      & U_x(l_2,t) = 0, \\
      & U(-0, t) = U(+0, t), \\
      & a_1^2 U_{x}(-0, t) = a_2^2 U_{x}(+0, t). \\
    \end{aligned}
    \right.
  \end{equation*}
\end{frame}

\begin{frame}
  \frametitle{Разделение переменных}
  Пусть $ U(x,t)=\theta(t)\varphi(x) $, тогда
  \begin{equation*}
    \frac{\theta'}{\theta}=\frac{a^2\varphi''+w\varphi'}{\varphi}=s.
    \label{eq:1}
  \end{equation*}
  Константа $ s $ не может быть положительной, в противном случае функция $\theta(t)$ будет не ограничена. Ввиду этого, можно принять
  \begin{equation*}
    s=-\lambda^2.
    \label{eq:2}
  \end{equation*}
\end{frame}

\begin{frame}
  \frametitle{Составление задачи Штурма--Лиувилля}
  Получаем задачу Штурма--Лиувилля для $\varphi(x)$:
  \begin{equation*}
    a^2 \varphi''(x) + w \varphi'(x) + \lambda^2\varphi(x)=0 
  \end{equation*}
  с внешними граничными условиями и условиями сопряжения:
  \begin{equation*}
    \left\{
    \begin{aligned}
      & \varphi_1(-l_1) = 0, \\
      & \varphi_2'(l_2) = 0, \\
      & \varphi_1(0) = \varphi_2(0), \\
      & a_1^2 \varphi_1'(0) = a_2^2 \varphi_2'(0). \\
    \end{aligned}
    \right.
  \end{equation*}
\end{frame}

\begin{frame}
  \frametitle{Составление задачи Штурма--Лиувилля}
  Для того чтобы исключить из уравнения слагаемое $w\varphi'(x)$, производим замену 
  \begin{equation*}
    \varphi(x) = \psi(x) e^{- \mu x},\ \text{где}\ \mu=\mu(x) = \frac{w(x)}{2a(x)^2}.
  \end{equation*}
  В итоге получаем новое уравнение
  \begin{equation*}
    a^2\psi''(x) + (\lambda^2 - \mu^2a^2) \psi(x) = 0
    \label{eq:3}
  \end{equation*}
  и новые граничные условия
  \begin{equation*}
    \left\{
    \begin{aligned}
      & \psi_1(-l_1) = 0, \\
      & \psi_2'(l_2) - \mu\psi_2(l_2) = 0, \\
      & \psi_1(0) = \psi_2(0), \\
      & a_1^2(\psi_1'(0) - \mu\psi_1(0)) =  a_2^2(\psi_2'(0) - \mu\psi_2(0)). \\
    \end{aligned}
    \right.
    \label{}
  \end{equation*}
\end{frame}
\end{comment}

\begin{frame}
  \frametitle{Уравнение для собственных чисел}
%  Пусть, для определенности, $\mu_1 a_1 < \mu_2 a_2$.

  Уравнение имеет разный вид на трех промежутках:
  \begin{enumerate}
    \item $ 0 \le \lambda \le \mu_1a_1$
    \item $ \mu_1a_1 < \lambda \le \mu_2a_2 $
    \item $\mu_1a_1 < \lambda < \infty$
  \end{enumerate}
\end{frame}
\begin{frame}
  \frametitle{Уравнение для собственных чисел}
  Рассмотрим отрезок $ 0 \le \lambda \le \mu_1a_1$. Уравнение имеет вид: 
  \begin{equation*}
    \begin{aligned}
    & -\sh(\gamma_1(\lambda) l_1) ((a_2^2 \mu_2-a_1^2 \mu_1) (\gamma_2(\lambda) \ch(\gamma_2(\lambda) l_2)-\sh(\gamma_2(\lambda) l_2) \mu_2)+a_2^2 \gamma_2(\lambda)- \\
    & -(\gamma_2(\lambda) \sh(\gamma_2(\lambda) l_2)-\ch(\gamma_2(\lambda) l_2) \mu_2))-a_1^2 \gamma_1(\lambda) \ch(\gamma_1(\lambda) l_1) \times \\
    & \times (\gamma_2(\lambda) \ch(\gamma_2(\lambda) l_2)-\sh(\gamma_2(\lambda) l_2) \mu_2) = 0.
    \end{aligned}
  \end{equation*}
  Где
  \begin{equation*}
    \begin{aligned}
      \gamma_1 = \gamma_1(\lambda) = \sqrt{- \frac{\lambda^2}{a_1^2} + \mu_1}, \\
      \gamma_2 = \gamma_2(\lambda) = \sqrt{- \frac{\lambda^2}{a_2^2} + \mu_2}.
    \end{aligned}
  \end{equation*}
  Два оставшихся промежутка рассматриваются аналогично.
\end{frame}

\begin{frame}
  \frametitle{Общий вид собственных функций}
  Так же, как и собственные числа, в каждом их трех промежутков для $\lambda$, собственные функции будут иметь разный вид:
  \begin{enumerate}
    \item $ 0 \le \lambda \le \mu_1a_1$
    \item $ \mu_1a_1 < \lambda \le \mu_2a_2 $
    \item $\mu_1a_1 < \lambda < \infty$
  \end{enumerate}
\end{frame}

\begin{frame}
  \frametitle{Общий вид собственных функций}
    В первом промежутке $ 0 \le \lambda \le \mu_1a_1$ собственные функции равны
    \begin{equation*}
      \begin{aligned}
        \varphi_1(x) = e^{\mu x} \left( A_1 \ch \gamma_1x + B_1 \sh \gamma_1x \right),\\
        \varphi_2(x) = e^{\mu x} \left( A_2 \ch \gamma_2x + B_2 \sh \gamma_2x \right).
      \end{aligned}
    \end{equation*}
    Где
    \begin{equation*}
      \begin{aligned}
        & \gamma_1 = \gamma_1(\lambda) = \sqrt{- \frac{\lambda^2}{a_1^2} + \mu_1}, \\
        & \gamma_2 = \gamma_2(\lambda) = \sqrt{- \frac{\lambda^2}{a_2^2} + \mu_2}, \\
        & A_i, B_i = \const.
      \end{aligned}
    \end{equation*}
\end{frame}

\begin{frame}
  \frametitle{Общий вид собственных функций}
  Во втором и третьем промежутке соответственно собственные функции имеют следующий вид:
  \begin{columns}
    \column{0.5 \textwidth}
      \begin{equation*}
        \begin{aligned}
          & \varphi_1(x) = e^{\mu x} \left( A_1' \cos \gamma_1x + B_1' \sin \gamma_1x \right), \\
          & \varphi_2(x) = e^{\mu x} \left( A_2' \ch \gamma_2x + B_2' \sh \gamma_2x \right), \\
          & \gamma_1 = \gamma_1(\lambda) = \sqrt{\frac{\lambda^2}{a_1^2} - \mu_1}, \\
          & \gamma_2 = \gamma_2(\lambda) = \sqrt{- \frac{\lambda^2}{a_2^2} + \mu_2}.
        \end{aligned}
      \end{equation*}


    \column{0.5 \textwidth}
      \begin{equation*}
        \begin{aligned}
          & \varphi_1(x) = e^{\mu x} \left( A_1'' \cos \gamma_1x + B_1'' \sin \gamma_1x \right), \\
          & \varphi_2(x) = e^{\mu x} \left( A_2'' \cos \gamma_2x + B_2'' \sin \gamma_2x \right), \\
          & \gamma_1 = \gamma_1(\lambda) = \sqrt{\frac{\lambda^2}{a_1^2} - \mu_1}, \\
          & \gamma_2 = \gamma_2(\lambda) = \sqrt{\frac{\lambda^2}{a_2^2} - \mu_2}.
        \end{aligned}
      \end{equation*}
  \end{columns}
\end{frame}

\begin{comment}
\begin{frame}
  \frametitle{Нахождение веса}
  Собственные функции $\psi_n$ и $\psi_m$ ($n \ne m$) ортогональны с весом $\rho_\psi(x)$. Искомый вес находится из уравнения:
  \begin{equation*}
    \int \limits_{-l_1}^{l_2} \rho_\psi(x) \psi_n(x) \psi_m(x) dx = 0.
  \end{equation*}
  Можно показать что равенство выполняется при $\rho(x) \equiv 1$.
  Делая обратную замену для $\varphi = \psi e^{-\mu x}$, находим весовую функцию $\rho_\varphi$
  \begin{equation*}
    \rho_\varphi(x) = \rho(x) = e^{2 \mu x}.
  \end{equation*}
\end{frame}

\begin{frame}
  \frametitle{Проведение интегрального преобразования}
  Решение вспомогательной задачи представим в виде
  \begin{equation*}
    U(x,t)=\sum \limits_{n=1}^{\infty} \frac{\varphi_n(x) \theta_n(t)}{||\varphi_n||}.
  \end{equation*}
  Где
  \begin{equation*}
    \theta_n(t) = \int \limits_{-l_1}^{l_2} \rho U(x,t) \varphi_n(x) dx.
  \end{equation*}
\end{frame}

\begin{frame}
  \frametitle{Проведение интегрального преобразования}
  Для нахождения $\theta_n(t)$ необходимо провести интегральное преобразование обеих частей уравнения для $U(x,t)$:
  \begin{equation*}
    \begin{aligned}
      & \int \limits_{-l_1}^{l_2} \rho U_t \varphi_n dx = \overbrace{\int \limits_{-l_1}^{l_2} \rho w U_x \varphi_n dx}^{I_1} + \overbrace{\int \limits_{-l_1}^{l_2} \rho a^2 U_{xx} \varphi_n dx}^{I_2} +  \overbrace{\left( -\omega \sin \omega t \int \limits_{-l_1}^{l_2} \rho \varphi_n dx \right)}^{F(t)}  \Rightarrow \\
    & \Rightarrow \theta_n'(t) = I_1 + I_2 + F(t).
  \end{aligned}
  \end{equation*}
  После подстановок и преобразований, получим:
  \begin{equation*}
  \theta_n'(t) + \lambda_n^2 \theta_n(t) = F(t).
  \end{equation*}

\end{frame}
\end{comment}

\begin{frame}
  \frametitle{Нахождение временной составляющей решения}
  Уравнение для $\theta_n(t)$:
  \begin{equation*}
    \left\{  
    \begin{aligned}
      & \theta_n'(t) + \lambda_n^2 \theta_n(t) = F(t), \\
      & \theta_n (0) = 0.
    \end{aligned}
    \right.
  \end{equation*}
  Это линейное дифференциальное уравнение первого порядка, решение которого хорошо известно:
  \begin{equation*}
    \theta_n(t) = A\cdot\frac{\lambda_n^2 \sin \omega t - \omega \cos \omega t + \omega e^{-\lambda_n^2 t}}{\omega^2 + \lambda_n^4},\ \text{где}\ A=-\omega\int\limits_{-l_1}^{l_2}\rho(x)\varphi_n(x)dx.
  \end{equation*}
\end{frame}

\begin{frame}
  \frametitle{Выражение для результата}
    %Производим обратную замену: $u(x,t) = U(x,t) + (1 - \cos \omega t)$. Тогда окончательное решение задачи имеет вид:
    Окончательное решение задачи имеет вид:
    \begin{equation*}
      u(x,t)= 1 - \cos \omega t + \sum \limits_{n=1}^{\infty} \frac{\varphi_n(x) \theta_n(t)}{||\varphi_n||}.
    \end{equation*}

\end{frame}

\begin{frame}
  \frametitle{Сложность вычислений}
    Выберем следующие значения параметров:
    \begin{equation*}
      l_1=1,\ l_2=19,\ a_1^2=10^{-3},\ a_2^2=10^{-4},\ w_1=7.8\cdot 10^{-4},\ w_2=w_1 \cdot 0.8,
    \end{equation*}
    корнем первого уравнения для собственных чисел будет являться $\lambda=8 \cdot 10^{-29}$.
    Для его отыскания использовалась система компьютерной алгебры Pari/GP, 
    которая позволяет работать с числами произвольной точности.
\end{frame}

\begin{frame}
  \frametitle{Сложность вычислений}
  \begin{figure}[H]
   \includegraphics[width=11cm]{full.eps}
   \caption{Левая часть трансцендентного уравнения относительно $\lambda$}
  \end{figure}
\end{frame}

\begin{comment}
\begin{frame}
 \frametitle{Анимация процесса}
 Выберем значения параметров следующим образом:
  \begin{equation*}
    l_1=3,\ l_2=5,\ a_1=2,\ a_2=4,\ w_1=0.1,\ w_2=w_1 \cdot 0.8.
  \end{equation*}
 \begin{center}
   \animategraphics[autoplay,loop,height=5cm]{100}{a1/animation-one-}{1}{200} 
 \end{center}
\end{frame}


\begin{frame}
 \frametitle{Анимация процесса}
 Выберем значения параметров следующим образом:
  \begin{equation*}
    l_1=3,\ l_2=1,\ a_1=4,\ a_2=1,\ w_1=0.05,\ w_2=w_1 \cdot 0.8.
  \end{equation*}
 \begin{center}
   \animategraphics[autoplay,loop,height=5cm]{100}{a2/animation-two-}{1}{200} 
 \end{center}
\end{frame}
\end{comment}

\begin{frame}
 \frametitle{Анимация процесса}
 Сравним процесс с конвекцией (слева) и без конвекции (справа):
 \begin{center}
   \animategraphics[autoplay,loop,height=4cm]{100}{a3/animation-four-}{1}{200} 
 \end{center}
\end{frame}

\begin{frame}
  \frametitle{Заключение}
  \begin{itemize}
    \item Получено аналитическое решение задачи конвективной диффузии. Решение разбито на этапы и структурировано
    \item Показан вид искомой функции на конкретных примерах
    \item Рассмотрены трудности, которые могут возникнуть в процессе нахождения собственных значений
    \item Указанная схема может применяться в более сложных телах больших размерностей
  \end{itemize}
\end{frame}

\begin{frame}
  \frametitle{Заключение}
  \begin{center}
    Спасибо за внимание.

    Готов ответить на ваши вопросы.
  \end{center}
\end{frame}
\end{document}
