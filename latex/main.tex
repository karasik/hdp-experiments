\documentclass[12pt, a4paper]{article}
\usepackage{amsmath, amsfonts}
\usepackage[utf8]{inputenc}
\usepackage[russian]{babel}
\usepackage{graphicx}
\usepackage{wrapfig}
\usepackage{float}
\usepackage{geometry}
\usepackage[indentfirst,compact,topmarks,calcwidth,pagestyles]{titlesec}
\usepackage{verbatim}
\usepackage{titletoc}
\usepackage{cmap}
\usepackage{hyperref}
\textheight=24cm
\textwidth=16cm
\oddsidemargin=5mm
\evensidemargin=-5mm
\marginparwidth=36pt
\topmargin=-1cm
\footnotesep=3ex
\raggedbottom
\tolerance 3000
\clubpenalty=10000
\widowpenalty=10000
\usepackage[T2A]{fontenc}

\begin{document}
  \section{Введение}
  	Платформы микроблогов стали очень популярным способом размещения данных в Сети. В них можно найти сообщения пользователей практически на любую тему, начиная стихийными бедствиями и заканчивая рейтингами музыкальных исполнителей. Правильная обработка доступной информации --- нетривиальная задача, которая имеет множество областей применения.
  	
	Отслеживание сообщений о стихийных бедствиях в реальном времени поможет вовремя организовать спасательные операции и сохранить жизни людей\cite{nuggets}. Руководствуясь сообщениями пользователей микроблогов можно судить о популярности товаров и вовремя принимать экономически целесообразные решения. Можно делать предположения о рейтингах политиков и эффективности рекламы на основании информации в микроблогах. Помимо перечисленных способов применения доступной информации в микроблоггинговых платформах можно привести множество других.
	
	В данной работе в дальнейшем будет рассматриваться сервис микроблогов Twitter. В нем помимо текстовой информации можно публиковать фото и видео, что так же может быть использовано при анализе. В доступном наборе данных можно проводить анализ разных сущностей, эта работа посвящена выявлению событий среди потоков информации. Поскольку трактовка событий в сообщениях микроблогов может быть субъективной, выделим несколько свойств, которыми должно обладать событие.
	
	Событие в первую очередь должно быть чем-то аномальным на фоне остальных данных. Например это рез
	
  \section{Описание задачи}
  Цель данной работы --- разработать алгоритм, который по входным данным будет строить набор событий. В качестве данных были использованы сообщения пользователей Twitter с 4 июня 2013 года по 31 июня 2013 года, содержащие в себе хэштег \#texas. Прежде чем перейти к описанию искомого алгоритма, рассмотрим некоторые существующие подходы к данной задаче.
  \section{Полученные результаты}
  \section{Заключение}

\begin{thebibliography}{9}
	\bibitem{nuggets}
	Extracting Information Nuggets from Disaster-Related Messages in Social Media,
	Imran, Elbassuoni, Castillo, Diaz and Meier,
	2013.
\end{thebibliography}
  
\end{document}
