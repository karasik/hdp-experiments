\documentclass[12pt, a4paper]{article}
\usepackage{amsmath, amsfonts}
\usepackage[utf8]{inputenc}
\usepackage[russian]{babel}
\usepackage{graphicx}
\usepackage{wrapfig}
\usepackage{float}
\usepackage{geometry}
\usepackage[indentfirst,compact,topmarks,calcwidth,pagestyles]{titlesec}
\usepackage{verbatim}
\usepackage{titletoc}
\usepackage{cmap}
\usepackage{hyperref}
\textheight=24cm
\textwidth=16cm
\oddsidemargin=5mm
\evensidemargin=-5mm
\marginparwidth=36pt
\topmargin=-1cm
\footnotesep=3ex
\raggedbottom
\tolerance 3000
\clubpenalty=10000
\widowpenalty=10000
\usepackage[T2A]{fontenc}

\begin{document}
  \section{Введение}
  	В современном мире информация быстро устаревает, поэтому способы вовремя находить нужные данные --- постоянный объект для исследований. Одним из направлений в этой области является извлечение данных из микроблоггинговых платформ.
  	
  	Платформы микроблогов стали очень популярным способом размещения данных в Сети. В них можно найти сообщения пользователей практически на любую тему, начиная стихийными бедствиями и заканчивая рейтингами музыкальных исполнителей. Правильная обработка доступной информации --- нетривиальная задача, которая имеет множество областей применения. Последние несколько лет эта тема активно исследуется во многих университетах мира.
  	
	Отслеживание сообщений о стихийных бедствиях в реальном времени поможет вовремя организовать спасательные операции и сохранить жизни людей\cite{nuggets}. Руководствуясь сообщениями пользователей микроблогов можно судить о популярности товаров и вовремя принимать экономически целесообразные решения. Можно делать предположения о рейтингах политических деятелей и эффективности рекламы на основании информации в микроблогах. Помимо перечисленных способов применения доступной информации в микроблоггинговых платформах можно привести множество других.
	
	В данной работе в дальнейшем будет рассматриваться сервис микроблогов Твиттер\footnote{http://www.twitter.com}. В нем помимо текстовой информации можно публиковать фото, видео и геотеги, что так же может быть использовано при анализе, но в этой работе не рассматривается. В доступном наборе данных можно проводить анализ разных сущностей, эта работа посвящена выявлению событий среди потоков информации. Поскольку трактовка событий в сообщениях микроблогов может быть субъективной, выделим несколько свойств, которыми характеризуется событие.
	
	Событие в первую очередь должно быть чем-то аномальным на фоне остальных данных. Оно определяется резким изменением частотных характеристик некоторых слов в сообщениях. События в микроблогах носят взрывной характер, в течении нескольких часов частоты релевантных слов возрастают в десятки раз и так же быстро опускаются до нормального уровня. Примером события могут быть: стихийное бедствие, выход спорного законопроекта на резонансную тему, получение фильмом награды на кинофестивале.
	
	Исторически подходы к описанной задаче менялись, в следующем разделе будет рассмотрена формальная постановка задачи и эволюция способов ее решения.
	
  \section{Описание задачи}
	Цель данной работы состоит из нескольких частей:
\begin{itemize}
\item исследовать существующие подходы по извлечению событий из сети Твиттер,
\item исследовать возможность применения иерархического процесса Дирихле для решения описанной задачи,
\item разработать метод для извлечения событий на основе иерархоческого процесса Дирихле,
\item продемонстрировать работу алгоритма на реальных данных.
\end{itemize}	  
  
  Объект изучения этой работы --- алгоритм, который по входным данным строит множество событий. В качестве данных для задач подобного рода служит мультимножество документов (сообщений) $\left\{D_i \,\vert\, i \in \overline{1,n} \right\}$. Будем считать что каждый документ имеет временную метку. Документ $D_i$ определяется как упорядоченный набор слов $ \left\{ w_j \,\vert\, w_j \in D_i \right\} $. При этом слова в документах принадлежат некоторому словарю $V$. 
  
  Событие --- некоторая сущность, которая характеризуется временем возникновения и ключевыми словами. Оно вызывает резкий подъем частотных характеристик некоторых слов. Событием может быть футбольный матч и музыкальный концерт. В социальной сети Твиттер есть темы, которые всегда популярны. Например это сообщения с ключевыми словами iphone и ipad. Но такие сообщения нельзя считать событиями. Также событиями нельзя считать еженедельные пятничные сообщения о конце рабочей недели\cite{waim13}.
  
  Особенности социальной сети Твиттер состоят в следующем:
  \begin{itemize}
  \item короткие сообщения (до 140 символов),
  \item наличие шума и ошибок,
  \item большая плотность сообщений.
  \item взрывной характер событий,
  \end{itemize}
  
  \section{Обзор существующих подходов}
  Самым
  
  
  \section{Полученные результаты}
  В качестве данных были использованы сообщения пользователей Twitter с 4 июня 2013 года по 31 июня 2013 года, содержащие в себе хэштег \#texas. Всего корпус включает порядка 240 тысяч сообщений и 1.5 миллиона слов. 
  \section{Заключение}

\begin{thebibliography}{9}
	\bibitem{nuggets}
	Extracting Information Nuggets from Disaster-Related Messages in Social Media,
	Imran, Elbassuoni, Castillo, Diaz and Meier,
	2013.
	\bibitem{waim13}
	Real Time Event Detection in Twitter,
	Xun Wang, Feida Zhu, Jing Jiang, Sujian Li,
	2011.
\end{thebibliography}
  
\end{document}
